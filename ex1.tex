\documentclass[11pt]{article}
\usepackage{enumerate}
\usepackage{mathrsfs}
\usepackage{amsmath, amsfonts, amsthm, amssymb}
\setlength{\parindent}{0pt}

\begin{document}

\section*{Exercise 1.1}

\textit{If $\mu(\cdot)$ and $v(\cdot)$ are negligible functions then show that $\mu(\cdot) \cdot v(\cdot)$ is a negligible function
} \\

Assumptions:
\begin{align*}
        &\mu(\cdot) negligible \\
\iff &\forall c_1 \in \mathbb{Z}^{+}, \exists n_{0_1} \in \mathbb{Z}^{+} s.t. \forall n_1 \geq n_{0_1}, \mu(n_1) < n_1^{-c_1} \\
\\
&v(\cdot) negligible \\
\iff &\forall c_2 \in \mathbb{Z}^{+}, \exists n_{0_2} \in \mathbb{Z}^{+} s.t. \forall n_2 \geq n_{0_2}, v(n_2) < n_2^{-c_2}
\end{align*}

Want to show $f(\cdot) = \mu(\cdot)\cdot v(\cdot) negligible$, aka:
$$\forall c \in \mathbb{Z}^{+}, \exists n_0' \in \mathbb{Z}^{+} s.t. \forall n \geq n_0', f(n) < n^{-c}$$

Consider these two functions multiplied together. Notice that inequalities are preserved when multiplied by positive numbers. Also that we can keep unify the $c_1$ and $c_2$ into $c$ (because that's what a similar proof in notes does).

\begin{align*}
\mu(\cdot) \cdot v(\cdot) = &\forall c \in \mathbb{Z}^{+}, \exists n_{0_1},n_{0_2} \in \mathbb{Z}^{+} s.t. \\
  &\forall n_1 \geq n_{0_1}, n_2 \geq n_{0_2} \\
  &\mu(n_1)\cdot v(n_2) < n_1^{-c}\cdot n_2^{-c} \\
\end{align*}

Let $n_0' = max(n_{0_1}, n_{0_2})$ \\
Note that $n_0' \geq n_{0_1} \text{and}\ n_0' \geq n_{0_2}$. Therefore:

\begin{align*}
\forall n \geq n_0', \mu(n)\cdot v(n) &< n^{-c}\cdot n^{-c} = n^{-2c} < n^{-c}
\end{align*}

Since $f(n) = \mu(n)\cdot v(n)$, we've shown:

$$\forall c \in \mathbb{Z}^{+}, \exists n_0' \in \mathbb{Z}^{+} s.t. \forall n \geq n_0', f(n) < n^{-c}$$

Thus $f(\cdot) negligible$ \qed

\section*{Exercise 1.2}

\textit{If $\mu(\cdot)$ is a negligible function and $f(\cdot)$ is a function polynomial in its input then show that $\mu(f(\cdot))$ is a negligible function.
} \\

Assumptions:
\begin{align*}
        &\mu(\cdot) negligible \\
\iff &\forall c \in \mathbb{Z}^{+}, \exists n_{0_1} \in \mathbb{Z}^{+} s.t. \forall n \geq n_{0_1}, \mu(n) < n^{-c} \\
\\
&f(\cdot) poly \\
\iff &\forall d \in \mathbb{N}, \exists C,k \in \mathbb{N}\ s.t.\ \forall m > k, f(m) \leq Cm^{d}
\end{align*}

Want to show:
\begin{align*}
        &\mu(f(\cdot)) negligible \\
\iff &\forall c \in \mathbb{Z}^{+}, \exists n_0 \in \mathbb{Z}^{+} s.t. \forall n \geq n_0, \mu(f(n)) < n^{-c} \\
\iff &\forall d \in \mathbb{N}, \exists C,k \in \mathbb{N}\ s.t.\  \forall c \in \mathbb{Z}^{+}, \exists n_0 \in \mathbb{Z}^{+} s.t. \forall n \geq n_0, \mu(Cn^d) < n^{-c} \\
\end{align*}

Since we know $\mu(\cdot) negligible$, we know:
$$
\forall d \in \mathbb{N}, \exists C,k \in \mathbb{N}\ s.t.\  \forall c \in \mathbb{Z}^{+}, \exists n_{0_1} \in \mathbb{Z}^{+} s.t. \forall n \geq n_0, \mu(Cn^d) < (Cn^d)^{-c}
$$

So it suffices to choose a $n_0$ and show $(Cm^d)^{-c} \leq n^{-c}$, $\forall n \geq n_0$ and $\forall m \geq n_{0_1}$. \\

If $(Cm^d)^{-c} \leq n^{-c}$ is true $\forall m,n \geq$ some lower bound  then it's true at the lower bound. \\

Thus it suffices to choose a $n_0$ and show $(C{n_{0_1}}^d)^{-c} \leq {n_0}^{-c}$ \\
Let $n_0 = (C{n_{0_1}}^d)^{-c}$ \\
$(C{n_{0_1}}^d)^{-c} \leq (C{n_{0_1}}^d)^{-c}$ \\

Thus:
$$
\forall d \in \mathbb{N}, \exists C,k \in \mathbb{N}\ s.t.\  \forall c \in \mathbb{Z}^{+}, \exists n_0 \in \mathbb{Z}^{+} s.t. \forall n \geq n_0, \mu(Cn^d) < (Cn^d)^{-c} \leq n^{-c}
$$ \qed

\section*{Exercise 1.3}

\textit{Prove that the existence of one-way functions implies $P \neq NP$} \\

Assume for the sake of contradiction $P = NP$ and assume that one-way functions exist. We will show that $P = NP$ actually implies that one-way functions don't exist. \\

This means if something is easy (poly time) to verify it's easy to compute. \\

The definition of a one-way function is that the probability that an adversary $\mathcal{A}$ can invert a function $f$ invoked on some input $x$ uniformly at random is negligible w.r.t the size of $x$. \\

Verifying that some $y$ is a pre-image to a function $f$ that runs in poly time is now easy since $P = NP$. Easy means poly time. \\

Recall that negligible means that some function grows slower than an inverse polynomial. \\

An adversary $\mathcal{A}$ can now use the $P = NP$ oracle to find a preimage for $f$ deterministically in poly time. The probability that this will happen is 1. This is larger than an inverse polynomial. So one-way functions do not exist.

Thus we have a contradiction.

A bit hand-wavy, but you get the idea. \qed

\section*{Exercise 1.4}

\textit{Prove that there is no one-way function $f : \{0,1\}^n \rightarrow \{0,1\}^{\lfloor log_2n \rfloor}$} \\

Recall that a one-way function states that the probability an adverary $\mathcal{A}$ can find the preimage to a function is for some uniformly random chosen $x$ is negligible.

If the range of the function is only $\lfloor log_2n \rfloor$ many bits, then guessing randomly will be correct with probability $2^{-\lfloor log_2n \rfloor}$ basically probability $\frac{1}{n}$. This function is not negligible: When $c = 2$, $n^{-1} > n^{-2}$. \qed


\end{document}

